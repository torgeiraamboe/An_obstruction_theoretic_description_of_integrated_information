\section{Introduction}

Integrated information theory is a mathematical description of the existing consciousness of a system, developed mainly by Giulio Tononi and his collaborators, \citeme. The theory is centered around the computation of $\Phi$, which is a measure of the \emph{integrated information} in the system. Informally, a system $S$ has integrated information, captured by $\Phi(S)>0$, if the information generated by the entire system's cause-effect structure is greater than the information generated by the combined cause-effect structure of partitions of the system. The formal mathematical framework for computing $\Phi$ is rigorous and technical, and uses detailed features of probability theory and measure theory, see \citeme
% IIT 3 and 4

In the present paper we wish to study integrated information through the lens of obstruction theory, which is a mathematical framework studying certain obstruction classes that measures whether a property can be extended through defined levels of sub-structure. They live in mathematical objects called \emph{cohomology groups}, which are abstract formulations of the amount of holes a system has in different dimensions. Obstruction theory was originally developed by Stiefel and Whitney to prove the non-existence of certain vector-fields on manifolds \citeme, but has later been applied in a plethora of mathematical contexts \citeme. 

The conceptual similarity between these two concepts is also extendable to a formal description, which is the aim of this short paper. Building on work by Kleiner and Tull on a categorical description of integrated information theory, see \citeme, we assign a simplicial complex $N^\bullet$ to an IIT system $S$, and use this to build cohomology groups valued in a presheaf that incorporates the cause-effect dynamics of $S$. Intuitively, we wish to build a mathematical structure whose $n$-dimensional holes represent $n$-th order integration, or causal connectedness. We then define a natural obstruction class from any attempted partition $\psi$ of $S$ into subsystems, which we show implies properties for IIT. In particular, if there is a non-trivial obstruction class, then the system has integrated information. These classes will detect whether the global cause-effect dynamics of $S$ can be ``glued together'' from the local dynamics of its decompositions. This gluing is described by the presheaf, and the failure to glue is measured by the cohomology. 

The upshot of this work is that one can study integrated information from a new lens, and apply fundamental ideas from algebraic topology and homological algebra to study phenomena in consciousness. In particular, our theory gives a much more general approach to integrated information, that side steps a lot of technicalities. We expect that our methods can be combined with topological data analysis, see \citeme for an introduction, to locate minimum partitions (MIP) and connect integrated information to new developments in the relationship between topology and information theory, see for example \citeme and \citeme. 
%Li, Information Topology (arXiv)
%Varley, Mediano, Patania, Bongard, “The topology of synergy” (PLOS Comp Bio, 2025)
%Edelsbrunner–Virk–Wagner, “Topological Data Analysis in Information Space”

We make no claim surrounding the validity of IIT to explain consciousness and the properties of experience, only to connect some of its mathematical properties to other mathematical concepts known to measure similar features. We view this work not as a new standard for IIT, but as a starting point to explore some of the rich mathematical structures that related to it. 